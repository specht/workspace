% !TEX program = lualatex
\documentclass[11pt,a4paper]{article}

\usepackage[margin=20mm]{geometry}
\usepackage{fontspec}
\usepackage{microtype}
\usepackage{xcolor}
\usepackage{tabularx}
\usepackage{enumitem}
\usepackage{listings}
\usepackage{tabto}

% ---------- Mathematik ----------
% Default-Mathefont (robust, passt zu fast allem)
\usepackage{unicode-math}
% \setmainfont{XCharter}
% \setmathfont{XCharter Math}

\pagestyle{empty}

% ---------- Layout ----------
\setlength{\parindent}{0pt}
\setlength{\parskip}{6pt}

% ---------- Beispielsätze (LaTeX-sicher: \$ statt $) ----------
\newcommand{\abc}{abcdefghijklmnopqrstuvwxyz}
\newcommand{\ligs}{\ ff fi fl ffi ffl}
\newcommand{\abctextrm}{Text roman:  \abc\ligs}
\newcommand{\abctextit}{Text italic: \ \ \textit{\abc\ligs}}
\newcommand{\abcmathit}{Math italic: \ $\abc$}
\newcommand{\formulas}{%
  $\displaystyle
   \int_{0}^{1}\! x^{2}\,\mathrm{d}x = \dfrac{1}{3} 
   \qquad
   \sum_{k=0}^{\infty} \frac{1}{k^{2}}=\frac{\pi^{2}}{6}
   \qquad
   \mathcal{ABC}\ \mathfrak{ABC}\ \mathbb{CNQR}$}
\newcommand{\doall}{\abctextrm\par\abctextit\par\abcmathit\par\formulas}

\newcommand{\pangram}{Franz jagt im komplett verwahrlosten Taxi quer durch Bayern.}
\newcommand{\intl}{Falsches Üben von Xylophonmusik quält jeden größeren Zwerg.}
\newcommand{\greekintl}{Ξεσκεπάζω την ψυχοφθόρα βδελυγμία.}

\newcommand{\numbers}{0123456789\hspace{0.5em}—\hspace{0.5em} € £ \$ ¥ © ®\hspace{0.5em}·\hspace{0.5em}}
\newcommand{\sampletext}{%
  Normal: \tabto{3.5cm}\intl\\
  {\em Kursiv:}\tabto{3.5cm}{\em \intl}\\
  {\bfseries Fett:}\tabto{3.5cm}{\bfseries \intl}\\
  {\bfseries\em Fett + Kursiv:}\tabto{3.5cm}{\bfseries\em \intl}\\
  Symbole:\tabto{3.5cm}\numbers\ligs
}
\newcommand{\sampletextgreek}{%
  Normal: \tabto{3.5cm}\greekintl\\
  {\em Kursiv:}\tabto{3.5cm}{\em \greekintl}\\
  {\bfseries Fett:}\tabto{3.5cm}{\bfseries \greekintl}\\
  {\bfseries\em Fett + Kursiv:}\tabto{3.5cm}{\bfseries\em \greekintl}\\
  Polytonie:\tabto{3.5cm}Ἀγάπη, Ὀδύσσεια, Παιδεία\\
  Symbole:\tabto{3.5cm}\numbers\ligs
}

\newcommand{\headline}[2]{%
  {\large\bfseries #1}\hfill{\small\ttfamily #2}\par
}

% showfont(Titel, Font-Familienname wie in fc-list)
\newcommand{\showfont}[2]{%
  \clearpage
  \vspace{5pt}
  \hrule
  \vspace{5pt}
  \headline{\fontspec{#2}{#1}}{#2}
  {%
    \fontspec{#2}\sampletext\par
  }%
  \vspace{5pt}
  \hrule
  \vspace{5pt}
}

% showfont(Titel, Font-Familienname wie in fc-list)
\newcommand{\showfontgreek}[2]{%
  \clearpage
  \vspace{5pt}
  \hrule
  \vspace{5pt}
  \headline{\fontspec{#2}{#1}}{#2}
  {%
    \fontspec{#2}\sampletextgreek\par
  }%
  \vspace{5pt}
  \hrule
  \vspace{5pt}
}

% showfont(Titel, Textfont, Mathefont) — lokal gekoppelt
\newcommand{\showfontmath}[3]{%
  \clearpage
  \vspace{5pt}
  \hrule
  \vspace{5pt}
  \headline{\fontspec{#2}{#1}}{#2 / #3}
  {%
    \fontspec{#2}%
    \setmathfont{#3}%
    \sampletext\hspace{1em} $\mathcal{ABC}\ \mathfrak{ABC}\ \mathbb{CNQR}$\par
    \[
      \int_0^\pi \textrm{\setmainfont{#2}sin}(x)\,dx = 2
      \qquad
      \sum_{k=1}^{n} k = \frac{n(n+1)}{2}
    \]
  }%
  \vspace{5pt}
  \hrule
  \vspace{5pt}
}

% ---------- Listings ----------
\lstset{
  columns=fullflexible,
  frame=single,
  breaklines=true,
  showstringspaces=false
}

\title{Schriftarten-Galerie für LuaLaTeX}
\author{}
\date{}

\begin{document}
% \maketitle

% \tableofcontents
% \bigskip
% \hrule
% \vspace{10pt}

% ============================================================
% \section{Textschrift + passende Mathematik}
% In LaTeX sind Textschrift und Mathe-Schrift getrennt. Einige Schriftfamilien liefern jedoch eine passende Mathe-Schrift mit, was das Gesamtbild harmonischer macht. Im Folgenden sind einige Beispiele aufgeführt.

\showfontmath{Latin Modern Roman}{Latin Modern Roman}{Latin Modern Math}
\showfontmath{XCharter}{XCharter}{XCharter Math}
\showfontmath{Stix Two}{Stix Two Text}{Stix Two Math}
\showfontmath{TeX Gyre Pagella + Pagella Math}{TeX Gyre Pagella}{TeX Gyre Pagella Math}
\showfontmath{TeX Gyre Termes + Termes Math}{TeX Gyre Termes}{TeX Gyre Termes Math}
\showfontmath{TeX Gyre Schola + Schola Math}{TeX Gyre Schola}{TeX Gyre Schola Math}
\showfontmath{TeX Gyre Bonum + Bonum Math}{TeX Gyre Bonum}{TeX Gyre Bonum Math}

% \hrule
% \vspace{10pt}

% ============================================================
% \section{Klassiker (typisches LaTeX-Aussehen)}
\showfont{Latin Modern Roman}{Latin Modern Roman}
\showfont{Latin Modern Sans}{Latin Modern Sans}
\showfont{Latin Modern Mono}{Latin Modern Mono}

\showfont{CMU Serif}{CMU Serif}
\showfont{CMU Sans Serif}{CMU Sans Serif}
\showfont{CMU Typewriter Text}{CMU Typewriter Text}

% ============================================================
% \section{Buch- und Textschriften (für längere Texte)}
\showfont{Nimbus Roman (Times-ähnlich)}{Nimbus Roman}
\showfont{Vollkorn}{Vollkorn}

% ============================================================
% \section{Moderne serifenlose Schriften (Arbeitsblätter, Präsentationen)}
\showfont{Inter}{Inter}
\showfont{IBM Plex Sans}{IBM Plex Sans}
\showfont{Ubuntu}{Ubuntu}

% ============================================================
% \section{Word-ähnliche Schriften (gut für Kompatibilität)}
\showfont{Liberation Serif (ähnlich Times New Roman)}{Liberation Serif}
\showfont{Liberation Sans (ähnlich Arial)}{Liberation Sans}
% \showfont{Caladea (ähnlich Cambria)}{Caladea}
\showfont{Carlito (ähnlich Calibri)}{Carlito}

% ============================================================
% \section{Monospace-Schriften (Code, Terminal, Informatik)}
{\small \showfont{JetBrains Mono}{JetBrains Mono} }
\showfont{Anonymous Pro}{Anonymous Pro}
{\small \showfont{IBM Plex Mono}{IBM Plex Mono}}

% \bigskip
% \headline{Beispiel: Quellcode (gleicher Code, andere Schrift)}{listings}

% \textbf{JetBrains Mono}
% {\lstset{basicstyle=\small\fontspec{JetBrains Mono}}
% \begin{lstlisting}
% def begruessung(name):
%     return "Hallo, " + name + "!"

% for n in ["Ada", "Linus", "Grace"]:
%     print(begruessung(n))
% \end{lstlisting}}

% \textbf{Fira Code}
% {\lstset{basicstyle=\small\fontspec{Fira Code}}
% \begin{lstlisting}
% if (a <= b && b != 0) {
%   const prozent = (a / b) * 100;
%   console.log("Ergebnis: " + prozent + "%");
% }
% \end{lstlisting}}

% \textbf{Hack}
% {\lstset{basicstyle=\small\fontspec{Hack}}
% \begin{lstlisting}
% SELECT titel, jahr
% FROM filme
% WHERE jahr >= 2000
% ORDER BY titel;
% \end{lstlisting}}

% ============================================================
% \section{Gut lesbar / kreativ (für Überschriften, Plakate)}
\showfont{Atkinson Hyperlegible (sehr gut lesbar)}{Atkinson Hyperlegible}
\showfont{Comic Neue (locker, aber sauber)}{Comic Neue}
\showfont{Montserrat}{Montserrat}
\showfont{Comfortaa}{Comfortaa}

% ============================================================
% \section{Griechische Schriftarten}
\showfontgreek{GFS Bodoni}{GFS Bodoni}
\showfontgreek{GFS Didot}{GFS Didot}
\showfontgreek{GFS Neohellenic}{GFS Neohellenic}


% \section{Alle Schriftgrößen}

% {\tiny\intl}

% {\scriptsize\intl}

% {\footnotesize\intl}

% {\small\intl}

% {\normalsize\intl}

% {\large\intl}

% {\Large\intl}

% {\LARGE\intl}

% {\huge\intl}

% {\Huge\intl}

% \vspace{10pt}
% \hrule
% \vspace{8pt}

% \section*{Merksätze}
% \textbf{1) Schrift für normalen Text setzen:}
% \begin{verbatim}
% \usepackage{fontspec}
% \setmainfont{Inter}
% \end{verbatim}

% \textbf{2) Monospace-Schrift (für Code) setzen:}
% \begin{verbatim}
% \setmonofont{JetBrains Mono}
% \end{verbatim}

% \textbf{3) Mathe-Schrift setzen (optional):}
% \begin{verbatim}
% \usepackage{unicode-math}
% \setmathfont{STIX Two Math}
% \end{verbatim}

% \textbf{4) Text + passende Mathe-Schrift (TeX Gyre):}
% \begin{verbatim}
% \setmainfont{TeX Gyre Pagella}
% \setmathfont{TeX Gyre Pagella Math}
% \end{verbatim}

\end{document}
