\documentclass[11pt,a4paper]{article}
% ↑ Dokumentklasse: ein einfacher Artikel
%   11pt = Schriftgröße, a4paper = Papierformat

\usepackage{fontspec}
% ↑ Ermöglicht die Nutzung moderner Schriftarten

\setmainfont{Latin Modern Roman}
% ↑ Setzt die Standardschrift für den Text

\begin{document}
% ↑ Ab hier beginnt der sichtbare Inhalt des Dokuments

\section{Hallo LaTeX}
% ↑ Eine Überschrift (LaTeX nummeriert sie automatisch)

Das ist ein kurzer Text.
LaTeX kümmert sich automatisch um Abstände und Zeilenumbrüche.

\textbf{Dieser Text ist fett}, \textit{und dieser ist kursiv}.
% ↑ Formatierung per Befehl, nicht per Maus
% bf: boldface, it: italic

\end{document}
% ↑ Ende des Dokuments